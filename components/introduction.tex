\section{はじめに}

How to Reviseは,筑波大学マルチエージェントシステム研究室の論文に関する非公式指南書である.
任意の人間によって書かれ,任意の人間によって修正される.
指導教員とのディスカッションの中で,新たに得た発見や知見があれば積極的に加筆ならびに修正されたい.

論文を書く中で,とくにreviseの過程でこの指南書を参考にしてほしい.
誰かが行った過ちを同じように繰り返さないよう,この書には悪い例も積極的に記載する.

論文を修正しながら,指導教員から得たアドバイスをもとにこの本をさらに加筆してほしい.
多くのノウハウが集まることによって,研究室としての論文の質がさらに向上していく.
それだけでなく,誰かが書いたことを積極的に修正してほしい.
必ずしもそれが正しいわけではないからである.
また,文章構成を積極的に修正してほしい.
新しく図に関する節が必要と感じたらそれらを作成し,全体構成を変更してほしい.
決まった構成は少しずつ腐っていき,負債になるためである.
常に内容を修正し変化を取り入れる必要がある.


